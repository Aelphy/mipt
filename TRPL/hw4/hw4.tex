 \documentclass[12pt]{article}
\usepackage[T2A]{fontenc}
\usepackage[utf8]{inputenc}        % Кодировка входного документа;
                                    % при необходимости, вместо cp1251
                                    % можно указать cp866 (Alt-кодировка
                                    % DOS) или koi8-r.

\usepackage[english,russian]{babel} % Включение русификации, русских и
                                    % английских стилей и переносов
%%\usepackage{a4}
%%\usepackage{moreverb}
\usepackage{indentfirst}
\usepackage{amsmath,amsfonts,amsthm,amssymb,amsbsy,amstext,amscd,amsxtra,multicol}
\usepackage{verbatim}
\usepackage{tikz} %Рисование автоматов
\usetikzlibrary{automata,positioning}
\usepackage{multicol} %Несколько колонок
\usepackage{graphicx}
\usepackage[colorlinks,urlcolor=blue]{hyperref}
\usepackage[stable]{footmisc}

%% \voffset-5mm
%% \def\baselinestretch{1.44}
\renewcommand{\theequation}{\arabic{equation}}
\def\hm#1{#1\nobreak\discretionary{}{\hbox{$#1$}}{}}
\newtheorem{Lemma}{Лемма}
\theoremstyle{definiton}
\newtheorem{Remark}{Замечание}
%%\newtheorem{Def}{Определение}
\newtheorem{Claim}{Утверждение}
\newtheorem{Cor}{Следствие}
\newtheorem{Theorem}{Теорема}
\theoremstyle{definition}
\newtheorem{Example}{Пример}
\newtheorem*{known}{Теорема}
\def\proofname{Доказательство}
\theoremstyle{definition}
\newtheorem{Def}{Определение}

%% \newenvironment{Example} % имя окружения
%% {\par\noindent{\bf Пример.}} % команды для \begin
%% {\hfill$\scriptstyle\qed$} % команды для \end






%\date{22 июня 2011 г.}
\let\leq\leqslant
\let\geq\geqslant
\def\MT{\mathrm{MT}}
%Обозначения ``ажуром''
\def\BB{\mathbb B}
\def\CC{\mathbb C}
\def\RR{\mathbb R}
\def\SS{\mathbb S}
\def\ZZ{\mathbb Z}
\def\NN{\mathbb N}
\def\FF{\mathbb F}
%греческие буквы
\let\epsilon\varepsilon
\let\es\varnothing
\let\eps\varepsilon
\let\al\alpha
\let\sg\sigma
\let\ga\gamma
\let\ph\varphi
\let\om\omega
\let\ld\lambda
\let\Ld\Lambda
\let\vk\varkappa
\let\Om\Omega
\def\abstractname{}

\def\R{{\cal R}}
\def\A{{\cal A}}
\def\B{{\cal B}}
\def\C{{\cal C}}
\def\D{{\cal D}}
\let\w\omega

%классы сложности
\def\REG{{\mathsf{REG}}}
\def\CFL{{\mathsf{CFL}}}
\newcounter{problem}
\newcounter{uproblem}
\newcounter{subproblem}
\def\pr{\medskip\noindent\stepcounter{problem}{\bf \theproblem .  }\setcounter{subproblem}{0} }
\def\prstar{\medskip\noindent\stepcounter{problem}{\bf $\theproblem^*$\negthickspace.  }\setcounter{subproblem}{0} }
\def\prpfrom[#1]{\medskip\noindent\stepcounter{problem}{\bf Задача \theproblem~(№#1 из задания).  }\setcounter{subproblem}{0} }
\def\prp{\medskip\noindent\stepcounter{problem}{\bf Задача \theproblem .  }\setcounter{subproblem}{0} }
\def\prpstar{\medskip\noindent\stepcounter{problem}{\bf Задача $\bf\theproblem^*$\negthickspace.  }\setcounter{subproblem}{0} }
\def\prdag{\medskip\noindent\stepcounter{problem}{\bf Задача $\theproblem^{^\dagger}$\negthickspace\,.  }\setcounter{subproblem}{0} }
\def\upr{\medskip\noindent\stepcounter{uproblem}{\bf Упражнение \theuproblem .  }\setcounter{subproblem}{0} }
%\def\prp{\vspace{5pt}\stepcounter{problem}{\bf Задача \theproblem .  } }
%\def\prs{\vspace{5pt}\stepcounter{problem}{\bf \theproblem .*   }
\def\prsub{\medskip\noindent\stepcounter{subproblem}{\rm \thesubproblem .  } }
\def\prsubstar{\medskip\noindent\stepcounter{subproblem}{\rm $\thesubproblem^*$\negthickspace.  } }
%прочее
\def\quotient{\backslash\negthickspace\sim}
\begin{document}
\centerline{\LARGE Задание 4}

\medskip

\begin{center}
	{\Large Замкнутость регулярных языков, теорема Майхилла-Нероуда и минимальные автоматы}
\end{center}

\begin{center}
	{Усвяцов Михаилб 176б}
\end{center}

\bigskip
		
		\prdag Доказать, что регулярные языки замкнуты относительно взятия морфизма.

		\prdag Верно ли, что для любого языка $L$ и любого морфизма $\varphi : \Sigma^* \to \Sigma^*$

		\prsub язык $\varphi(\varphi^{-1}(L))$ совпадает с $L$?

		\prsub язык $\varphi^{-1}(\varphi(L))$ совпадает с $L$?

		\prsub $\varphi(\varphi^{-1}(L)) \stackrel{?}{=} \varphi^{-1}(\varphi(L))$

		\medskip
		
		\prdag Доказать, что регулярные языки замкнуты относительно операции взятия обратного морфизма.

\prp Доказать, что для языка $L$ выполняется лемма о накачке, но он не является регулярным. Обозначим за ${\rm PRIMES}$ множество простых чисел. Напомним, что $A^+ = AA^*$.
$$ L = b^* \cup \{ab^{\,p}\,|\, p \in {\rm PRIMES} \} \cup aa^+b^*.  $$

\medskip

\prp К языку $L_1$ добавили конечный язык $R$ и получили язык $L$ ($L = L_1\cup R$). Язык $L$ оказался регулярным. Верно ли, что язык $L_1$ мог быть нерегулярным?

\bigskip

\prp Язык $L$ задан автоматом $\A$. Построить минимальный автомат для языка $L$.

\begin{tikzpicture}[shorten >=1pt,node distance=2cm,on grid,auto,initial text=]
  %\draw[help lines] (0,0) grid (3,2);
  \node[state ,initial] (q_0) 					 {$q_0$};
  \node[state, accepting]		    (q_1) [ right = of q_0 ] {$q_1$};
  \node[state]          (q_2) [ right = of q_1] {$q_2$};
  \node[state, accepting]           (q_3) [ right = of q_2] {$q_3$};

  \path[->] 
		(q_0)	edge		node	{$a$}	(q_1)
				edge [bend left] node {$b$}	(q_3)
		(q_1)
				edge		node	{$b$}	(q_2)
		(q_2)
				edge		node	{$b$}	(q_3)
				edge [bend left] node {$a$}	(q_1);
 \end{tikzpicture}
\medskip

\prp Постройте минимальный автомат для языка $\bar L$ из предыдущей задачи.

\bigskip

\prp Найдите все классы эквивалентности Майхилла-Нероуда для языка $\Sigma^*ab\Sigma^*$ и постройте по ним ДКА.


\end{document}