 \documentclass[12pt]{article}
\usepackage[T2A]{fontenc}
\usepackage[utf8]{inputenc}        % Кодировка входного документа;
                                    % при необходимости, вместо cp1251
                                    % можно указать cp866 (Alt-кодировка
                                    % DOS) или koi8-r.

\usepackage[english,russian]{babel} % Включение русификации, русских и
                                    % английских стилей и переносов
%%\usepackage{a4}
%%\usepackage{moreverb}
\usepackage{amsmath,amsfonts,amsthm,amssymb,amsbsy,amstext,amscd,amsxtra,multicol}
\usepackage{indentfirst}
\usepackage{verbatim}
\usepackage{tikz} %Рисование автоматов
\usetikzlibrary{automata,positioning}
\usepackage{multicol} %Несколько колонок
\usepackage{graphicx}
\usepackage[colorlinks,urlcolor=blue]{hyperref}
\usepackage[stable]{footmisc}

%% \voffset-5mm
%% \def\baselinestretch{1.44}
\renewcommand{\theequation}{\arabic{equation}}
\def\hm#1{#1\nobreak\discretionary{}{\hbox{$#1$}}{}}
\newtheorem{Lemma}{Лемма}
\theoremstyle{definiton}
\newtheorem{Remark}{Замечание}
%%\newtheorem{Def}{Определение}
\newtheorem{Claim}{Утверждение}
\newtheorem{Cor}{Следствие}
\newtheorem{Theorem}{Теорема}
\theoremstyle{definition}
\newtheorem{Example}{Пример}
\newtheorem*{known}{Теорема}
\def\proofname{Доказательство}
\theoremstyle{definition}
\newtheorem{Def}{Определение}

%% \newenvironment{Example} % имя окружения
%% {\par\noindent{\bf Пример.}} % команды для \begin
%% {\hfill$\scriptstyle\qed$} % команды для \end






%\date{22 июня 2011 г.}
\let\leq\leqslant
\let\geq\geqslant
\def\MT{\mathrm{MT}}
%Обозначения ``ажуром''
\def\BB{\mathbb B}
\def\CC{\mathbb C}
\def\RR{\mathbb R}
\def\SS{\mathbb S}
\def\ZZ{\mathbb Z}
\def\NN{\mathbb N}
\def\FF{\mathbb F}
%греческие буквы
\let\epsilon\varepsilon
\let\es\emptyset
\let\eps\varepsilon
\let\al\alpha
\let\sg\sigma
\let\ga\gamma
\let\ph\varphi
\let\om\omega
\let\ld\lambda
\let\Ld\Lambda
\let\vk\varkappa
\let\Om\Omega
\def\abstractname{}

\def\R{{\cal R}}
\def\A{{\cal A}}
\def\B{{\cal B}}
\def\C{{\cal C}}
\def\D{{\cal D}}
\let\w\omega

%классы сложности
\def\REG{{\mathsf{REG}}}
\def\CFL{{\mathsf{CFL}}}
\newcounter{problem}
\newcounter{uproblem}
\newcounter{subproblem}
\def\pr{\medskip\noindent\stepcounter{problem}{\bf \theproblem .  }\setcounter{subproblem}{0} }
\def\prp{\medskip\noindent\stepcounter{problem}{\bf Задача \theproblem .  }\setcounter{subproblem}{0} }
\def\prstar{\medskip\noindent\stepcounter{problem}{\bf Задача $\theproblem^*$ .  }\setcounter{subproblem}{0} }
\def\prdag{\medskip\noindent\stepcounter{problem}{\bf Задача $\theproblem^\dagger$ .  }\setcounter{subproblem}{0} }
\def\upr{\medskip\noindent\stepcounter{uproblem}{\bf Упражнение \theuproblem .  }\setcounter{subproblem}{0} }
%\def\prp{\vspace{5pt}\stepcounter{problem}{\bf Задача \theproblem .  } }
%\def\prs{\vspace{5pt}\stepcounter{problem}{\bf \theproblem .*   }
\def\prsub{\medskip\noindent\stepcounter{subproblem}{\rm \thesubproblem .  } }
%прочее
\def\quotient{\backslash\negthickspace\sim}
\begin{document}
	\centerline{\LARGE Задание 1}

	\medskip

	\centerline{\Large Регулярные языки и автоматы}
	\centerline{Усвяцов Михаил, группа 176б}

	\bigskip

	\section{Задачи предбазового уровня}

	\prp 

	\prsub $\{a, aa\}\cdot\{b,bb\} = \{ab, abb, aab, aabb\}$. По определению операции "конкатенация".

	\prsub $\{a, aa\}+\{b,bb\} = \{ a, aa, b, bb \}$. По определению операции "Объединение".

	\prsub $\{a, aa\}\times\{b,bb\} = \{ (a,b), (a, bb), (aa, b), (aa, bb) \}$. По определению операции "Декартово произведение".

	\prsub $ ((aa|b)^*(a|bb)^*)^* = ? $\\
	Заметим, что:\\
	$ ((aa|b)^*(a|bb)^*)^* = \{ \{ x, y \}^* | x \in \{ aa,b \}^*, y \in \{ a,bb \}^* \} = A $\\
	А теперь рассмотрим следующее множество  B:\\
	$ B = \{ a,b \}^* $\\
	Очевидно, что $ A \subset B $.\\
	Так как подмножеством A является множество $ \{ \{ x, y \}^* | x \in \{ b \}^*, y \in \{ a \}^* \} $, отсюда следует, что  $ B \subset A $. Из этих фактов следует, что A = B.
	\prsub 
	$ \{ a^{3n} | \, n>0\} \cap \{ a^{5n+1} | n \geq 0\}^* =?$\\
	$ \{ a^{5n + 1} | n \geq 0 \}^* = \{ a^{(5n + 1)k} | n \geq 0, k \geq 0 \} $\\
	$ \{ a^{3n} | \, n>0\} \subset \{ a^{(5n + 1)k} | n \geq 0, k \geq 0 \} \Longrightarrow 
	 \\ \{ a^{3n} | \, n>0\} \cap \{ a^{5n+1} | n \geq 0\}^*  =  \{ a^{3n} | \, n>0\} $

	\prsub  $\emptyset \cap \{\eps\} =?$\\
	$ A \cap B = \{ x | x \in A$ и $x \in B \} \Longrightarrow \emptyset \cap \{\eps\} = \emptyset$


	\section{Задачи базового уровня}

	\prp Записать регулярное выражение для языка $L =\Sigma^*\setminus{\{ (a|b)^*ab(a|b)^* \}}$. Доказать, что язык порождённый регулярным выражением совпадает с $L$.

	Рассмотрим регулярное выражение $ (b^*a^*) $. Оно соответсвует множеству $ A = \{ b^*a^* \} $. Множество $ B = {\{ (a|b)^*ab(a|b)^* \}} $ содержит все слова, содержащие ab, следовательно, L содержит все слова, которые не содержат ab, так как состоит из всех возможных слов над алфавитом {а, b}, которые не входят в B, следовательно  $L \subset A$. A задает множество всех слов над алфавитом {а, b}, которые не содержат ab. Следовательно, $A \subset L$ $\Longrightarrow$ A = L.
	
	\medskip
	

	\prp\\
	\begin{multicols}{2}

	\centerline{Автомат $\A:$}
	\vspace{10pt}

		\begin{tikzpicture}[shorten >=1pt,node distance=2cm,on grid,auto,initial text=]
		  %\draw[help lines] (0,0) grid (3,2);
		  \node[state,initial]  (q_0)                      {$q_0$};
		  \node[state, accepting]          (q_1) [above right = of q_0] {$q_1$};
		  \node[state]          (q_2) [below right =of q_1] {$q_2$};
		 % \node[state,accepting](q_3) [below right=of q_1] {$q_3$};
		  \path[->] 
			(q_0)	edge	[loop below]	node	{0}	()
					edge	[bend left] 	node	{1}	(q_1)
		 	(q_1)	edge	[bend left]		node	{1} (q_0)
					edge	[bend left]		node	{0}	(q_2)
			(q_2)	edge	[bend left]		node	{0}	(q_1)
					edge	[loop below]	node	{1}	();
		\end{tikzpicture}

	\centerline{Автомат $\B:$}
	\vspace{10pt}

		\begin{tikzpicture}[shorten >=1pt,node distance=2cm,on grid,auto,initial text=]
		  %\draw[help lines] (0,0) grid (3,2);
		  \node[state,initial]  (q_0)                      {$q_0$};
		  \node[state, accepting]          (q_1) [above right = of q_0] {$q_1$};
		  \node[state]          (q_2) [below right =of q_1] {$q_2$};
		 % \node[state,accepting](q_3) [below right=of q_1] {$q_3$};
		  \path[->] 
			(q_0)	edge	[loop below]	node	{0}	()
					edge	[bend left] 	node	{1}	(q_1)
		 	(q_1)	edge	[bend left]		node	{0} (q_0)
					edge	[bend left]		node	{0}	(q_2)
			(q_2)	edge	[bend left]		node	{0}	(q_1)
					edge	[loop below]	node	{1}	();
		\end{tikzpicture}

	\end{multicols}

	Для каждого автомата ответьте на следующие вопросы (1-2):

	\prsub Автомат задан через граф переходов. Запишите определение автомата в виде $(Q, \Sigma, \delta, q_0, F)$. Опишите элементы каждого множества 
	
	Автомат $\A:$\\
	$ Q = \{q_0, q_1, q_2 \} $\\
	$ \Sigma = \{ 0, 1 \} $\\
	$ q_0 = q_0 $\\
	$ \delta(q_0,0) = q_0\\
	   \delta(q_0, 1) = q_1\\
	   \delta(q_1, 0) = q_2\\
	   \delta(q_1, 1) = q_0\\
	   \delta(q_2, 0) = q_1\\
	   \delta(q_2, 1) = q_2\\
	$
	$ F = \{q_1\} $ \\
	
	Автомат $\B:$\\
	$ Q = \{q_0, q_1, q_2 \} $\\
	$ \Sigma = \{ 0, 1 \} $\\
	$ q_0 = q_0 $\\
	$ \delta(q_0,0) = q_0\\
		\delta(q_0, 1) = q_1\\
		\delta(q_1, 0) = \{q_0, q_2 \}\\
		\delta(q_2, 0) = q_1\\
		\delta(q_2, 1) = q_2\\
	$
	$ F = \{q_1\} $ \\

	\prsub Явлется ли автомат детерминированным?
	
	Автомат $\A:$
	Является, так как все функции переходов однозначны
		
	Автомат $\B:$
	Не является, так как существует неоднознаячная функция переходов

	\prsub Опишите последовательность конфигураций автомата $\A$ при обработке слова $w = 1101010$. Верно ли, что $w \in L(\A)$?
	
	$q_0 \longrightarrow q_1 \longrightarrow q_0 \longrightarrow q_0 \longrightarrow q_1 \longrightarrow q_2 \longrightarrow q_2 \longrightarrow q_1$
	
	Так как конечное состояние автомата является принимающим, то  $w \in L(\A)$.

	\prsub Принимает ли автомат $\B$ слово $v = 10001$?\\
	Да, принимает. Так как существует такая цепочка конфигурация, которая при данном слове приведет автомат в принимающее состояние.

	\prsub Укажите по одному слову, принадлежащему $L(\A)$, $L(\B)$ и по одному слову, не принадлежащее $L(\A)$, $L(\B)$. Все $4$ слова должны быть различными. \\
	$ 1101010 \in L(\A)$, $ 1101011 \notin L(\A)$\\
	$ 10001 \in L(\B)$, $ \epsilon \notin L(\B)$\\

	\prp
	
	Определим язык $L\subseteq \{a,b\}^*$ индуктивными правилами:\\
	({\em 1}) $\eps \in L$;\\ ({\em 2}) вместе с любым словом $x \in L$ в $L$ также 
	входят слова $xa, xaa, xabba$;\\ ({\em 3}) никаких других слов в $L$ нет.

	В язык $T\subseteq \{a,b\}^*$ входит пустое слово $\eps$ и ВСЕ 
	начинающиеся и заканчивающиеся буквой $a$ слова, в которых нет подслов
	``$aba$'' или ``$bbb$''. 
	Докажите или опровергните, что $L=T$. 
	

	\prsub Докажите или опровергните, что $L=T$. \\
	abbabba подходит языку T. Но не подходит языку L. Следовательно, языки неэквивалентны. 

	\prsub
	
	Рассмотрим ДКА A:

	\vspace{10pt}
	
	\begin{tikzpicture}[shorten >=1pt,node distance=3cm,on grid,auto,initial text=]
		\node[state,initial, accepting]  (q_0)                      {$q_0$};
		\node[state, accepting]          (q_1) [right = of q_0] {$q_1$};
		\node[state]          (q_2) [above left =of q_1] {$q_2$};
		\node[state]          (q_3) [above  =of q_2] {$q_3$};
		\node[state]          (q_4) [above =of q_1] {$q_4$};
		\path[->] 
		(q_0)	edge	[] 	                   node	{a}	(q_1)
				   edge    [bend left]                     node{b} (q_3)
		(q_1)	edge	[loop below]    node	{a}	()
		           edge    []                     node {b} (q_2)
		(q_2)	edge	[]    				   node	{a}	(q_3)
				   edge    []					  node {b} (q_4)
		(q_3)	edge	[loop above]	node  {a, b}	()
		(q_4)   edge	[]                     node	{a}	(q_1)
		           edge   []                        node	{b}	(q_3);
	\end{tikzpicture}
	
	T = $ (a(a|b)^*a | \epsilon | a) \setminus ((a|b)^*(aba|bbb)(a|b)^*) $\\
	База индукции:\\ $\epsilon \in T, L(A)$\\
	Предположение индукции:\\
	$ v \in (a|b)^*$, такое, что в $q_0: v \in {\epsilon}, q_1: v \in T / \{\epsilon\}, q_2: v \in T \{\epsilon\} b, q_3: v \in ((a(a|b)^*(aba|bbb)(a|b)^*)|b), q_4: v = T / \{\epsilon\} bb $\\
	Шаг индукции:\\
	Рассмотрим слова va и vb:\\
	$q_0: va \in T, L(A), vb \notin T, L(A)$\\
	$q_1: va \in T, L(A), vb \notin T, L(A)$\\
	$q_2: va \notin T, L(A), vb \notin T, L(A)$\\
	$q_3: va \notin T, L(A), vb \notin T, L(A)$\\
	$q_4: va \in T, L(A), vb \notin T, L(A)$\\
	Следовательно, все слова либо распознаются обоими языками, либо нераспознаются обоими, что означает, что множетсва T и L(A) равны.
\end{document}